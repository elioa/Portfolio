\documentclass{article}
\usepackage[italian]{babel}
\usepackage{geometry}
\RequirePackage[nomessages]{fp}
% Dimensioni derivate
\newcommand{\dimensioni}{%
% Dimensioni pagina
\FPeval{PageHeightiii}{\PageHeight/3}
\FPeval{PageHeightii}{\PageHeight-\PageHeight/3}
\FPeval{PageWidthiii}{\PageWidth/3}
\FPeval{PageWidthii}{\PageWidth-\PageWidth/3}
\FPeval{PageWidthAP}{round(\PageWidth:2)}
\FPeval{PageHeightAP}{round(\PageHeight:2)}
% Rientro Lettrine
\FPeval{Lettrine}{1/3}
% Indent
\FPeval{ParIndent}{\BaseLineSkip}
\FPeval{ParIndentAP}{round(\ParIndent:2)}
% Margini posizioni relative al bordo vicino
\FPeval{LeftMargin}{round(\xLeftMargin:2)}
\FPeval{RightMargin}{round(\PageWidth-\xRightMargin:2)}
\FPeval{BottomMargin}{round(\yBottomMargin:2)}
\FPeval{TopMargin}{round(\PageHeight-\yTopMargin:2)}
% Altezza e larghezza del testo
\FPeval{xyTextWidth}{\PageWidth-\LeftMargin-\RightMargin}
\FPeval{xyTextHeight}{\PageHeight-\BottomMargin-\TopMargin}
\FPeval{TextWidth}{round(\xyTextWidth:2)}
\FPeval{TextHeight}{round(\xyTextHeight:2)}
\FPeval{TextHeightiii}{\TextHeight/3}
\FPeval{TextHeightii}{\TextHeight-\TextHeight/3}
\FPeval{TextWidthiii}{\TextWidth/3}
\FPeval{TextWidthii}{\TextWidth-\TextWidth/3}
% Separazione della pagina
\FPeval{yFootSkip}{\BottomMargin/(3/2)}
\FPeval{FootSkip}{round(\BottomMargin-\yFootSkip:2)}
% Dimensioni dei font
\FPeval{FontSizePage}{\FontSize*2/3}
\FPeval{FontSizeTitle}{\FontSize*5/3}
\FPeval{FontSizeAuthor}{\FontSize*4/3}
\FPeval{BaseLineTitle}{\FontSizeTitle*1.2}
\FPeval{FontSizeAP}{round(\FontSize:2)}
\FPeval{FontSizePageAP}{round(\FontSizePage:2)}
\FPeval{FontSizeTitleAP}{round(\FontSizeTitle:2)}
\FPeval{FontSizeAuthorAP}{round(\FontSizeAuthor:2)}
\FPeval{BaseLineSkipAP}{round(\BaseLineSkip:2)}
\FPeval{BaseLineTitleAP}{round(\BaseLineTitle:2)}
% Titolo
\FPeval{TitoloWidth}{\PageWidth-\LeftMargin*2}
\FPeval{xTitolo}{\LeftMargin}
\FPeval{yTitolo}{\PageHeight/9}
\FPeval{xLogo}{\LeftMargin}
\FPeval{yLogo}{\PageHeight*8/9-\TextWidth*0.16180/2}
% Crediti
\FPeval{xCrediti}{\PageWidthiii}
\FPeval{CreditiWidth}{\PageWidthii-\LeftMargin}
% Immagine
\FPeval{yImmagine}{\PageHeight*3/9}
% Separazione della virgola finale
\FPeval{BaseLineSkipV}{5*\BaseLineSkip}
}

\newcommand{\dimePA}{%
\FPeval{PageWidth}{210}%
\FPeval{PageHeight}{297}%
\FProot\solution{385}{2}%
\FPeval{xLeftMargin}{1242150/14701-44100/14701*\solution}%
\FPeval{xRightMargin}{1845060/14701+44100/14701*\solution}%
\FPeval{yBottomMargin}{1756755/14701-62370/14701*\solution}%
\FPeval{yTopMargin}{2609442/14701+62370/14701*\solution}%
\FPeval{FontSize}{15}%
\FPeval{BaseLineSkip}{18.35}%
\dimensioni}

\dimePA
\geometry{paperwidth=\PageWidth mm,paperheight=\PageHeight mm,left=\LeftMargin mm,right=\RightMargin mm,top=\TopMargin mm,bottom=\BottomMargin mm,footskip=\FootSkip mm}
\setlength{\parindent}{0pt}

\usepackage{tikz}
\usepackage{graphicx}
\usepackage{calc}
\usepackage{adforn}
\usepackage[cmintegrals,cmbraces]{newtxmath}
\usepackage{ebgaramond-maths}
\usepackage[T1]{fontenc}
\usepackage[utf8]{inputenc}
\usepackage{anyfontsize}
\usepackage{changepage}


\begin{document}
\pagestyle{empty}

\null\hfill\begin{minipage}{0.4\linewidth}\raggedleft
\bgroup{\fontsize{20pt}{20pt}\selectfont
\sc Elio A. Farina}\\
\fontsize{10pt}{12pt}\selectfont
elio.farina@gmail.com
\egroup
\end{minipage}
\large

\vspace{28pt}
Titolo del seminario:\\[7pt]
{\LARGE La comunicazione in fisioterapia:\\ il rapporto fisioterapista--paziente--care giver}

\vspace{28pt}
\begin{adjustwidth}{42pt}{0pt}
\begin{center}
\textit{Possono diversi stili e strategie comunicative influenzare un trattamento fisioterapico?}\\[7pt]
{\huge \adfhangingflatleafleft}\\[7pt]
\end{center}

L'obiettivo di questo seminario è mostrare, attraverso il racconto dell'esperienza sul campo di un fisioterapista, come diversi atteggiamenti comunicativi nei confronti del paziente, del parente o del care-giver possono modificare in ogni direzione un trattamento fisioterapico, e come questi possano venir sfruttati a vantaggio del paziente o a vantaggio del riabilitatore.

Mostrerò una breve e non esaustiva analisi di come ottenere la fiducia del paziente per l'ottenimento dei risultati previsti negli obiettivi a breve e lungo termine, e come gestire il rapporto con il care giver o il parente, inserirlo nel trattamento come risorsa ed evitare che diventi un ostacolo.

Strategie verbali e non verbali come approccio al paziente con piccoli esempi pratici di vita lavorativa quotidiana.

Attraverso l'analisi di esperienze comuni e fatti di cronaca è possibile smascherare eventuali ciarlatani anche alla luce della analisi della letteratura scientifica.

Infine, come uno psicologo potrebbe intervenire all'interno di una équipe riabilitativa nella analisi di una terapia, nel rapporto tra il paziente e il trattamento o il parente e la struttura, il tutto dal punto di vista della mia esperienza come fisioterapista presso case di riposo e come assistenza domiciliare.
\end{adjustwidth}

\clearpage
\null\hfill\begin{minipage}{0.4\linewidth}\raggedleft
\bgroup{\fontsize{20pt}{20pt}\selectfont
\sc Elio A. Farina}\\
\fontsize{10pt}{12pt}\selectfont
elio.farina@gmail.com\\
badroomtales.com
\egroup
\end{minipage}
\large

\vspace{28pt}
Titolo del seminario:\\[7pt]
{\LARGE Tipografia:\\ come comunicare meglio con gli strumenti a propria disposizione}

\vspace{28pt}
\begin{adjustwidth}{42pt}{0pt}
\begin{center}
\textit{Il grassetto salverà il mondo?}\\[7pt]
{\huge \adfhangingflatleafleft}\\[7pt]
\end{center}

Nei nostri computer sono presenti due strumenti apparentemente molto potenti: un \textit{word processor} e uno \textit{slide show presentation program}. Basta avviarli che si è subito dentro l'azione. Ma senza una reale preparazione in termini di design è veramente possibile pensare di poter utilizzare quegli strumenti ottenendo il migliore degli obiettivi? L'idea di base, nel pensiero comune, è sì: io apro, schiaccio dei tasti, appaiono delle lettere. Ma la tipografia ha un piccolo difetto: \emph{funziona \emph{quando} non \emph{si vede}}. E se non si vede risulta difficile conoscerla a fondo. L'obiettivo di questo seminario è raccontare in breve come piccoli e all'apparenza insignificanti cambiamenti nella produzione di un qualsiasi testo possono modificare enormemente il risultato finale.

\end{adjustwidth}


\end{document}
