\initab
\hline
\aggiornamento{16 e 17 gennaio 2010, 16 ore, Cassano d'Adda ({\sc mi}).}{Kinesio Taping Italia.}{Kinesio Taping fundamentals and advanced (KT1 \&\ KT2).}{Dr. Ft. Frassine Stefano.}{introduzione del concetto e principi base del metodo (KT1), tecniche correttive (KT2), utilizzo del Kinesio Tex\reg\ Tape.}{14 (quattordici).}{\`e stato rilasciato un certificato di partecipazione.}
\hline
\hline
\aggiornamento{21 gennaio 2012, 8 ore, Cinisello Balsamo ({\sc mi}).}{Sinergia e sviluppo per AIFi Lombardia.}{Aggiornamenti giuridici e fiscali.}{Avv. Mauro Putignano, Comm. Chiara Orsatti, vari Dirigenti A.I.Fi..}{aggiornamenti in materia fiscale e giuridica riguardante la professione del fisioterapista.}{8 (otto).}{\`e stato rilasciato un certificato di partecipazione.}
\hline
\hline
\aggiornamento{20 e 21 febbraio 2012, 16 ore, Milano.}{Edi.Ermes.}{Tecniche di massaggio sportivo.}{Ft. Claudio Zimaglia.}{massaggio sportivo in tutte le fasi della gara e post infortunio.}{16 (sedici).}{\`e stato rilasciato un certificato di partecipazione.}
\hline
\hline
\aggiornamento{14--18 marzo e 6--10 giugno 2012, 35 ore a settimana, San Lazzaro di Savena~({\sc bo}).}{Format sas per ETGOM e AIFi Emilia Romagna.}{Corso di specializzazione in Medicina Ortopedica Cyriax.}{Ft. Maurizio Leone, Dr. Giuseppe Ridulfo.}{{\em Prima settimana} dal 14 al 18 marzo: introduzione alla Medicina Ortopedica Cyriax, esame obiettivo, lesioni e trattamento di ginocchio, caviglia e rachide lombare. {\em Seconda settimana} dal 6 al 10 giugno: esame obiettivo, lesioni e trattamento di caviglia, gomito, spalla, polso e rachide cervicale. 
%{\em Terza settimana} dal 10 al 14 di ottobre: in programma. {\em Quarta settimana} dal 30 gennaio al 3 febbraio 2013: in programma.
}
{50 (cinquanta) a settimana per un totale di 100 (duecento).}{\`e stato rilasciato un certificato di partecipazione per ogni settimana effettuata.
}
\hline
\hline
\aggiornamento{18 e 19 giugno 2012, 16 ore, Milano.}{Edi.Ermes.}{Bendaggio funzionale.}{Ft. Claudio Zimaglia.}{indicazioni e pratica del bendaggio funzionale.}{16 (sedici).}{\`e stato rilasciato un certificato di partecipazione.}
\hline
\hline
\aggiornamento{25 settembre e 17 dicembre 2012 sessione plenaria, 10 e 22 ottobre, 5 e 26 novembre 2012 sessioni in gruppi, Monza, Auditorium {\sc asl mb}.}{{\sc asl mb}.}{Operare nelle cure domiciliari in Lombardia: nuove sfide possibili.}{Dr.ssa Cicoletti Diletta, Dr. Di Ci\`o Francesco.}{nuovi scenari nelle cure domiciliari in Regione Lombardia.}{19,50 (diciannove,50).}{\`e stato rilasciato un certificato di partecipazione.}
\hline
\hline
\aggiornamento{16 marzo 2013, Hotel Habitat, Viale Como, 2, Giussano ({\sc mb}).}{Ikos {\sc srl}.}{I traumi dello sport. Approccio chirurgico o riabilitativo. La prevenzione e
la nutrizione.}{Docenti vari; coordinatori scientifici Prof.~Giuseppe Peretti e Dott.~Piero Ciampi}{Approccio chirurgico e riabilitativo ai traumi sportivi.}{8 (otto).}{\`e stato rilasciato un certificato di partecipazione.}
\hline
\hline
\aggiornamento{dal 12 al 19 ottobre 2013, dal 9 al 14 dicembre 2013 (in programma), {\sc u.o.} Medicina Riabilitativa, Via Carlo Alberto, 25, Bellano ({\sc lc}).}{Ufficio Formazione e Sviluppo, Azienda Ospedaliera della Provincia di Lecco.}{Corso {\sc Ibita} livello base, ``Valutazione e trattamento dell'adulto con disturbi neurologici. Concetto Bobath''}{Ft. Luca Cesana.}{Concetto Bobath.}{50 (cinquanta).}{.}
\hline
\end{tabularx}

