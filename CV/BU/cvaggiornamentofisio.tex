\paragrafo{Corsi 2010}{2010}
\aggiornamento{Bendaggio -- Kinesio Taping}{16 e 17 gennaio 2010, 16 ore, Cassano d'Adda ({\sc mi}).}{Kinesio Taping Italia.}{Kinesio Taping fundamentals and advanced (KT1 \&\ KT2).}{Dr. Ft. Frassine Stefano.}{introduzione del concetto e principi base del metodo (KT1), tecniche correttive (KT2), utilizzo del Kinesio Tex\reg\ Tape.}{14 (quattordici).}

\vspace{1.5em}
\paragrafo{Corsi 2012}{2012}
\aggiornamento{Varie -- Agg. giuridici e fiscali}{21 gennaio 2012, 8 ore, Cinisello Balsamo ({\sc mi}).}{Sinergia e sviluppo per AIFi Lombardia.}{Aggiornamenti giuridici e fiscali.}{Avv. Mauro Putignano, Comm. Chiara Orsatti, vari Dirigenti A.I.Fi..}{aggiornamenti in materia fiscale e giuridica riguardante la professione del fisioterapista.}{8 (otto).}

\aggiornamento{Massaggio -- Massaggio sportivo}{20 e 21 febbraio 2012, 16 ore, Milano.}{Edi.Ermes.}{Tecniche di massaggio sportivo.}{Ft. Claudio Zimaglia.}{massaggio sportivo in tutte le fasi della gara e post infortunio.}{16 (sedici).}

\aggiornamento{Ortopedico -- Ciriax}{14--18 marzo e 6--10 giugno 2012, 35 ore a settimana, San Lazzaro di Savena~({\sc bo}).}{Format sas per ETGOM e AIFi Emilia Romagna.}{Corso di specializzazione in Medicina Ortopedica Cyriax.}{Ft. Maurizio Leone, Dr. Giuseppe Ridulfo.}{{\em Prima settimana} dal 14 al 18 marzo: introduzione alla Medicina Ortopedica Cyriax, esame obiettivo, lesioni e trattamento di ginocchio, caviglia e rachide lombare. {\em Seconda settimana} dal 6 al 10 giugno: esame obiettivo, lesioni e trattamento di caviglia, gomito, spalla, polso e rachide cervicale.}{50 (cinquanta) a settimana per un totale di 100 (duecento).}

\aggiornamento{Bendaggio -- Bendaggio funzionale}{18 e 19 giugno 2012, 16 ore, Milano.}{Edi.Ermes.}{Bendaggio funzionale.}{Ft. Claudio Zimaglia.}{indicazioni e pratica del bendaggio funzionale.}{16 (sedici).}

\aggiornamento{Varie -- Cure domicialiari}{25 settembre e 17 dicembre 2012 sessione plenaria, 10 e 22 ottobre, 5 e 26 novembre 2012 sessioni in gruppi, Monza, Auditorium {\sc asl mb}.}{{\sc asl mb}.}{Operare nelle cure domiciliari in Lombardia: nuove sfide possibili.}{Dr.ssa Cicoletti Diletta, Dr. Di Ci\`o Francesco.}{nuovi scenari nelle cure domiciliari in Regione Lombardia.}{19,50 (diciannove,50).}

\vspace{1.5em}
\paragrafo{Corsi 2013}{2013}
\aggiornamento{Ortopedico -- Traumi dello sport}{16 marzo 2013, Hotel Habitat, Viale Como, 2, Giussano ({\sc mb}).}{Ikos {\sc srl}.}{I traumi dello sport. Approccio chirurgico o riabilitativo. La prevenzione e
la nutrizione.}{Docenti vari; coordinatori scientifici Prof.~Giuseppe Peretti e Dott.~Piero Ciampi}{Approccio chirurgico e riabilitativo ai traumi sportivi.}{8 (otto).}

\aggiornamento{Neutologico -- Bobath}{dal 12 al 19 ottobre 2013 e dal 9 al 14 dicembre 2013, {\sc u.o.} Medicina Riabilitativa, Via Carlo Alberto, 25, Bellano ({\sc lc}).}{Ufficio Formazione e Sviluppo, Azienda Ospedaliera della Provincia di Lecco.}{Corso {\sc Ibita} livello base, ``Valutazione e trattamento dell'adulto con disturbi neurologici. Concetto Bobath''}{Ft. Luca Cesana, coadiuvato da Ft. Giuseppe Cultrera e Ft. Cristina Capra.}{Concetto Bobath.}{50 (cinquanta).}

\vspace{1.5em}
\paragrafo{Corsi 2014}{2014}
\aggiornamento{EBM -- Scale di valutazione}{15 febbraio 2014, {\sc u.o.} Azienda Ospedaliera di Lecco, Via dell'Eremo 9/11, Lecco.}{Sinergia \&{} Sviluppo srl.}{Le scale di misura dell'outcome clinico in riabilitazione.}{Ft. Daniele Piscitelli.}{Applicazione nella pratica quotidiana dei principi e delle procedure dell'evidence based practice (ebm - ebn - ebp).}{4 (quattro).}


