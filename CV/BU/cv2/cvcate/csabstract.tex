\begin{center}
{\sc\Huge -- Pubblicazioni --}
\end{center}

\section*{}
\begin{tabular}{p{100pt}|p{400pt}}
\multicolumn{2}{l}{{\bf\Large Manoscritto in preparazione}}\\
\cline{1-1}
&\\[-5pt]
&{\bf ``Positioning bias in portraits and self-portraits''} (Suitner, C. \& Maass, A.).\\
\cline{2-2}
\multicolumn{2}{l}{}\\
\multicolumn{2}{l}{{\bf\large Abstract}}\\
\cline{1-1}
&\\[-5pt]
&The aim of the research project is to investigate the left vs. right orientation in visual arts, with particular emphasis on the positioning bias in portraits. Painters rarely draw their sitters fully facing the observer of the painting. Rather they tend to depict half prof\mbox{}iles that confer depth to the f\mbox{}igures. Painting a half-prof\mbox{}ile f\mbox{}igure implies a necessary choice of which of the two half faces should be dominant. Studies on this topic have generally shown that this is not a random choice, but that there are systematic biases that af\mbox{}fect the direction of portraits painting: the so called positioning bias (Chatterjee, 2002). The hypothesized explanations are tested in three studies: two on portraits (Study 1 and 2) and one on self-portraits (Study 3). The present research project replicated two f\mbox{}indings already reported in the literature, namely (a) the tendency to portray women sitters in a stronger left-ward position than male sitters (Study 1 and 2) and (b) the possibility that this gender bias is decreasing in more recent times (Study 2). More importantly, this research analyzes a number of new variables, including the role of the artist's gender (Study 1 and 3), the ef\mbox{}fect of the sitter's age (Study 1,2 and 3), and the ef\mbox{}fect of the sitter's social status (Study 2) on positioning bias. Study 1 showed that female artists did not follow the general leftward bias in their depictions. To the contrary, they showed a preference for rightward representations. The ef\mbox{}fect of artist's gender on the positioning bias is strictly linked to the ef\mbox{}fect of sitter's gender. In fact, female artists tended to portray females with a stronger rightward orientation than males. The ef\mbox{}fect of artist's gender in self-portrait (Study 3) was quite dif\mbox{}ferent from the artist's gender ef\mbox{}fect found in portraits (Study 1). Whereas it seemed f\mbox{}irst to parallel the ef\mbox{}fect of sitter's gender of both Study 1 and 2, with women represented facing left and men facing right, at a second analysis it was found to interact with time. The ef\mbox{}fect of sitter's age; it was tested in all three studies. In both Study 1 and 2 the descriptive data followed the direction predicted by the hypothesized curvilinear relation between age and spatial positioning, with young adults and old adults represented facing left and adults represented facing right. This curvilinear relation was statistically signif\mbox{}icant only in Study 1. Study 2 showed an increment of rightward bias associated to increasing social rank.\\
\cline{2-2}
\end{tabular}
