%%%%%%%%%%%%%%%%%%%%%%%%%%%%%%%%%%%%%%%%%%%%%%%%%%%%%%%%%%%%%%%%%%%%%
%%%%%%%%%%%%%%%%%%%%%%%%%%%%%%%%%%%%%%%%%%%%%%%%%%%%%%%%%%%%%%%%%%%%%
% 
% michael-cv.cls
% 
% LaTeX class template for Curriculum Vitae
% Version 1.0 (15/05/17)
% 
% Sponsor and maintainer Michael Rouba michael.rouba@roubacon.de
% Design by Andreas Wichmann wichmann@solution-scout.de
% Code by Elio. A. Farina elio.farina@gmail.com
% 
% Created under the Creative Commons "Namensnennung 3.0 DE" License.
% 
%%%%%%%%%%%%%%%%%%%%%%%%%%%%%%%%%%%%%%%%%%%%%%%%%%%%%%%%%%%%%%%%%%%%%
%%%%%%%%%%%%%%%%%%%%%%%%%%%%%%%%%%%%%%%%%%%%%%%%%%%%%%%%%%%%%%%%%%%%%
\documentclass[english]{michael-cv}
\usepackage[utf8]{inputenc}

\usepackage{booktabs}
\definecolor{marginins}{RGB}{0,54,68}
\definecolor{titlecins}{RGB}{253,89,0}
\definecolor{martexins}{RGB}{166,166,166}
\definecolor{martitins}{RGB}{217,217,217}
\definecolor{footerins}{RGB}{32,93,111}
\definecolor{maruleins}{gray}{0}
\definecolor{margininsg}{gray}{1}
\definecolor{maruleinsg}{gray}{0.3}
\definecolor{titlecinsg}{gray}{0.2}
\definecolor{martexinsg}{gray}{0.3}
\definecolor{martitinsg}{gray}{0.3}
\definecolor{footerinsg}{gray}{0.3}
\begin{document}

\renewcommand{\CVrightmargin}{150}
\renewcommand{\CVleftmargin}{50}

\renewcommand{\CVProfil}{CV Template}
\renewcommand{\CVCV}{Manual}

\renewcommand{\infoot}{michael-cv.cls}
\renewcommand{\outfoot}{Page~\thepage}
\renewcommand{\leftfoot}{\small Created under the Creative Commons ``Namensnennung 3.0 DE'' License.}

\newcommand{\titolo}[1]{\vspace{2ex}
\textcolor{titlec}{\Large\bf #1}\\[1ex]}

\CVname{Michael Rouba CV class}
\CVprof{Instruction manual}
\CVpict{logoEAF}
\CVtele{}
\CVcell{}
\CVmail{}
\CVmadd{}
\CVaddr{}
\CVinfo{}
\CVtitle

\begin{para}{}
\bf These are the instructions of michael-cv.cls, a class for your CV.
\end{para}

\vspace{4\baselineskip}
\entry{Credits}{Sponsor and maintainer}{Michael Rouba\hfill \textcolor{titlec}{michael.rouba@roubacon.de}}{}
\entry{\null}{Design by}{Andreas Wichmann\hfill \textcolor{titlec}{wichmann@solution-scout.de}}{}
\entry{\null}{Code by}{Elio. A. Farina\hfill \textcolor{titlec}{elio.farina@gmail.com}}{}
 
\entry{License}{Created under the}{Creative Commons ``Namensnennung 3.0 DE'' License.}{}

\vfill
\titolo{Class options}

\begin{para}{Class options: printable version}
You can create a coloured version of your curriculum by default.\\
If you want to switch all the colors off you can easily declare the \texttt{print} option:\\
\null\hspace{4ex}\verb+\documentclass[print]{michael-cv}+\\[2ex]
The \texttt{print} version turns all colours from RGB into gray scale.
\end{para}

\begin{para}{Class options: language}
Your document is set to be in German, but you can switch to English using \texttt{english} option:\\
\null\hspace{4ex}\verb+\documentclass[english]{michael-cv}+
\end{para}

\begin{para}{Class options: noborder}
The default version of the layout is with the picture with the same alignment of the rest of the text in the margin and with a black border. The \texttt{noborder} option\\
\null\hspace{4ex}\verb+\documentclass[noborder]{michael-cv}+\\[2ex]
turn off the border colour an move the picture next to it. In the \texttt{print} version the border remains gray.
\end{para}

\newpage
\titolo{CV commands}

\begin{para}{Your personal info commands}
At the begining of your document, soon after\\
\null\hspace{4ex}\verb+\begin{document}+\\[2ex]
you must write these commands, in order to set your personal info:\\[2ex]
\begin{tabular}{ll}
\toprule
Command			& \texttt{info} stands for\dots	\\
\midrule
\verb+\CVname{info}+	& Your full name	\\
\verb+\CVprof{info}+	& Your profile info	\\
\verb+\CVpict{info}+	& *Your picture file	\\
\verb+\CVtele{info}+	& *Telephone number	\\
\verb+\CVcell{info}+	& *Mobile phone number	\\
\verb+\CVmail{info}+	& *E-mail		\\
\verb+\CVmadd{info}+	& *Additional e-mail	\\
\verb+\CVinfo{info}+	& *Other infos		\\
\verb+\CVaddr{info}+	& *Adress info		\\
\bottomrule
\end{tabular}

\vspace{1ex}
You can write these commands sorted as you want. The commands marked by * are not mandatory: if your leave them empty, or you even never write them at all, \LaTeX\ won't print anything.

Soon after these commands, you must write\\
\null\hspace{4ex}\verb+\CVtitle+\\[1ex]
in order to print that commands in the right position:
\begin{itemize}
\item \texttt{CVname} will be the title of the document and the inner side of the right footer, where \texttt{CVaddr} is printed on the outer side;
\item \texttt{CVprof} info is the subtitle of the document;
\item The picture is set to be in the right margin, followed by all the other infos in this order:\\
\texttt{CVtele}, \texttt{CVcell}, \texttt{CVmail}, \texttt{CVmadd}, \texttt{CVinfo}.\\
You cannot change this order unless you change it in your class file.
\end{itemize}
\end{para}

\begin{para}{Writing paragraph}
To write the paragraph you can use the environment \texttt{para}:\\
\null\hspace{4ex}\verb+\begin{para}{title}+\\
\null\hspace{4ex}\verb+Your contents+\\
\null\hspace{4ex}\verb+\end{para}+\\[1ex]
This will print a paragraph as width as the text width with a distance from above of one base line space, like these ones I'm writing on. \texttt{title}, if present, will print the title of the paragraph in bold and coloured. If you need to write in \textbf{bold}, or \textit{italic}, just set the \LaTeX\ commands within the environment.
\end{para}

\newpage
\begin{para}{Writing CV entries}
To write the entries in a three columns environment there is just one command:\\
\null\hspace{4ex}\verb+\entry{title}{subtitle}{contents}{margin}+\\[1ex]
\verb+\entry+ command takes 4 mandatory argument. Some argument could be left empty leaving the \verb+{}+ blank, but you must write all 4 \verb+{}+.
\begin{itemize}
\item \texttt{title}: this prints the title in the first column in orange or gray.
\item \texttt{subtitle}: this prints the title in the second column in bold.
\item \texttt{contents}: this argument prints the contents in the second column.
\item \texttt{margin}: this argument prints the contents in the margin column. If \texttt{title} is present, they will appears white, or black in the printable version, and bolded. If \texttt{title} is not present, they will be printed in gray. If you want to print them in white but the \texttt{title} argument is not present, you must write \verb+\null+ instead of a title. If you want to write a list, or simply break the line, use the breakline command \verb+\\+ at the end of the word(s).
\end{itemize}
You can see some example below made by these commands.
\end{para}
\vfill
\begin{para}{}
With title and subtitle:\\
\verb+\entry{Title in orange}{Subtitle bolded}%+\\\verb+{Some contents}{Margin contents in white}+\\
\end{para}
\entry{Title in orange}{Subtitle bolded}{Some contents}{Margin contents in white}
\vfill
\begin{para}{}
With no title but subtitle:\\
\verb+\entry{}{Subtitle bolded}{Some contents\\ in two lines}%+\\\verb+{Margin contents\\ in gray}+\\
\end{para}
\entry{}{Subtitle bolded}{Some contents\\ in two lines}{Margin contents\\ in gray}
\vfill
\begin{para}{}
With title but no subtitle:\\
\verb+\entry{Title in orange}{}{Some contents and many more}%+\\\verb+{Margin contents in white}+\\
\end{para}
\entry{Title in orange}{}{Some contents and many more}{Margin contents in white}
\vfill
\begin{para}{}
\emph{Null} title, bolded subtitle and margin contents in white:\\
\verb+\entry{\null}{Subtitle bolded}%+\\\verb+{Some contents and many more}{Margin\\ contents\\ in white}+\\
\end{para}
\entry{\null}{Subtitle bolded}{Some contents and many more}{Margin\\ contents\\ in white}

\newpage
\begin{para}{No panic}
If this\\
\null\hspace{4ex}\verb+\entry{Title}{Subtitle}{A very long content that you don't know how to see it in one line}{Too\\ much\\ information\\ needed}+\\[1ex]
could drive you crazy, you can easily write the same command in multiple lines, as you can do with any \LaTeX\ command:\\
\null\hspace{4ex}\verb+\entry%+\\[-2ex]
\null\hspace{4ex}\verb+{Title}%+\\[-2ex]
\null\hspace{4ex}\verb+{Subtitle}%+\\[-2ex]
\null\hspace{4ex}\verb+{A very long content that you don't know how to see it in one line}%+\\[-2ex]
\null\hspace{4ex}\verb+{Too\\ much\\ information\\ needed}+\\
\end{para}

\entry{Title}{Subtitle}{A very long content that you don't know how to see it in one line}{Too\\ much\\ information\\ needed}

\titolo{Basic \LaTeX}

\begin{para}{Additional packages}
If you need additional packages you can write the command\\
\null\hspace{4ex}\verb+\usepackage{package}+\\[2ex]
in the preamble as usual.
\end{para}

\begin{para}{\LaTeX\ line and page breaking commands}
You can use basic \LaTeX\ layout commands like \verb+\\+ to break a line or \verb+\newpage+ to break a page.
\end{para}

\begin{para}{\LaTeX\ sectioning}
This class is based on \texttt{article} standard class. Sections are not set in this class, so if you'll use \verb+\section{}+ it will appear like a normal basic section.
\end{para}

\begin{para}{\LaTeX\ commands and environment}
You can use \texttt{itemize}, \texttt{enumerate} and \texttt{description} and (quite) all the other \LaTeX\ commands and environment. If you see any bug, please contact me.
\end{para}

\newpage
\titolo{Deep customizations}

\begin{para}{Customize your template}
You can easily customize your template using the \verb+\renewcommand+ with these already made commands:\\[1ex]
\begin{tabular}{ll}
\toprule
Command to be \emph{renewed}	& Default		\\
\midrule
\verb+\CVProfil+		& Profil		\\
\verb+\CVCV+			& Curriculum Vitae	\\
\verb+\CVTeleicon+		& \Telefon		\\
\verb+\CVMobiicon+		& \Mobilefone		\\
\verb+\CVmailicon+		& \MVAt			\\
\verb+\CVmaddicon+		& \MVAt			\\
\bottomrule
\end{tabular}

\vspace{1ex}
\begin{itemize}
\item \verb+\CVProfil+ is the right header in the first page;
\item \verb+\CVCV+ is the right header in all the others pages.
\end{itemize}

This manual is made by changing:\\
\null\hspace{4ex}\verb+\renewcommand{\CVProfil}{CV Template}+\\[-2ex]
\null\hspace{4ex}\verb+\renewcommand{\CVCV}{Manual}+\\[1ex]

Default icons use \texttt{marvosym} package; you can use any package for symbols you like, just remember to include it in your preamble.
\end{para}
\begin{para}{Customize the footer}
To change the footer you have just to \verb+\renewcommand+ these commands:\\[1ex]
\begin{tabular}{lll}
\toprule
Command to be \emph{renewed}	& Default	& Where	\\
\midrule
\verb+\infoot+			& The CVname	& inner right footer		\\
\verb+\outfoot+			& The CVaddress	& outer right footer			\\
\verb+\leftfoot+		& \emph{empty}	& left footer	\\
\bottomrule
\end{tabular}

\vspace{1ex}
This manual is made by changing:\\
\null\hspace{4ex}\verb+\renewcommand{\infoot}{michael-cv.cls}+\\[-2ex]
\null\hspace{4ex}\verb+\renewcommand{\outfoot}{Page~\thepage}+\\[-2ex]
\null\hspace{4ex}\verb+\renewcommand{\leftfoot}{\small Created under the Creative Commons...}+\\[1ex]
\end{para}

\newpage
\begin{para}{Customize the colours}
You can change the appearence using the \texttt{print} class option, or changing these colours using\\
\null\hspace{4ex}\verb+\definecolor{colour}{model}{color-spec}+\\[1ex]

In the next table you can see the colours in the coloured version:\\[1ex]
\begin{tabular}{lccc}
\toprule
Colour		& Model		& Color-spec	& Appearence	\\
\midrule
\verb+margin+	& RGB		& 0,54,68		& \textcolor{marginins}{Right margin}	\\
\verb+marule+	& gray		& 0			& \textcolor{maruleins}{Right margin border}			\\
\verb+titlec+	& RGB		& 253,89,0		& \textcolor{titlecins}{Titles}	\\
\verb+martit+	& RGB		& 217,217,217		& \textcolor{martitins}{Title in the margin}	\\
\verb+martex+	& RGB 		& 166,166,166		& \textcolor{martexins}{Text in the margin}	\\
\verb+footer+	& RGB		& 32,93,111		& \textcolor{footerins}{Footer}	\\
\bottomrule
\end{tabular}

\vspace{2ex}
In this table the same colours with \verb+print+ option:\\[1ex]
\begin{tabular}{lccc}
\toprule
Colour		& Model		& Color-spec	& Appearence	\\
\midrule
\verb+margin+	& gray 		& 1		& \textcolor{margininsg}{Right margin}	\\
\verb+marule+	& gray 		& 0.3 		& \textcolor{maruleinsg}{Right margin border}	\\
\verb+titlec+	& gray 		& 0.2		& \textcolor{titlecinsg}{Titles}	\\
\verb+martit+	& gray 		& 0.3		& \textcolor{martitinsg}{Title in the margin}	\\
\verb+martex+	& gray 		& 0.3		& \textcolor{martexinsg}{Text in the margin}	\\
\verb+footer+	& gray 		& 0.3		& \textcolor{footerinsg}{Footer}	\\
\bottomrule
\end{tabular}
\end{para}

\begin{para}{Customize align}
Your document is left aligned by default.\\
To customize the alignment write:\\[1ex]
\begin{tabular}{ll}
\toprule
Command				& Effect\\
\midrule
\verb+\newcommand{\CValign}{}+	& justified	\\
\verb+\newcommand{\CValign}{\RaggedRight}+	& left align	\\
\verb+\newcommand{\CValign}{\RaggedLeft}+	& right align	\\
\bottomrule
\end{tabular}
\end{para}

\newpage
\begin{para}{Customize the lengths}
You can customize the dimensions of any lenght an the paper with\\
\null\hspace{4ex}\verb+\renewcommand{thelength}{number(unit)}+\\[1ex]
For the * length in the next table you must express the unit of measure, for the rest the number must be set as absolute value of pt (points):\\[1ex]
\begin{tabular}{llp{140pt}}
\toprule
Length			& Default value	& Set	\\
\midrule
\verb+\CVtitparlen+	& $-$\verb+\baselineskip+	& *Distance after the paragraph title	\\
\verb+\CVpardis+	& \verb+\baselineskip+	& *Distance between two paragraphs or two entries	\\
\verb+\CVtsdisp+	& 1ex			& *Distance between CV title and subtitle	\\
\verb+\CVinfodis+	& 1ex			& *Distance between info below the picture		\\
\verb+\CVinmargin+	& 2ex			& *Inner margins (between elements)	\\
\verb+\CVleftmargin+	& 1/9 of paper width	& *Left margin	\\
\verb+\CVrightmargin+	& 1/3 of paper width	& *Right margin	\\
\verb+\CVtopmargin+	& 120			& Top margin	\\
\verb+\CVheadsep+	& 50			& Head sep	\\
\verb+\CVbottommargin+	& 50			& Bottom margin	\\
\verb+\CVentry+		& 1/3 of text width	& Length of the first column in \verb+\entry+	\\
\verb+\CVborder+	& 1			& Width of the margin border	\\
\verb+\CVtitlesep+	& \texttt{CVheadsep}$-$\texttt{CVtopmargin}$-$12~pt	& *Position of the title in the first page	\\
\verb+\CVpictwid+	& $-$6.5		& *Additional width of the picture$\dagger$	\\
\verb+\CVxinfobox+	& $-$1 & *Additional $x$ position of the infobox$\dagger$	\\
\verb+\CVyinfobox+	& \texttt{CVtopmargin}$-$20~pt		& *$y$ position of the infobox	\\
\verb+\CVpictalign+	& 0pt			& *Additional position in $x$ of the picture$\dagger$\\
\verb+\CVinfoalign+	& 0pt			& *Additional position in $x$ of the info$\dagger$\\
\bottomrule
\end{tabular}

\vspace{2ex}
The $\dagger$ length set an additional amount; \textit{i.e.}, the picture width is as width as the right margin, the value of \verb+\CVpictwid+ increase or decrease it by that amount.

\vspace{2ex}
This manual is made by changing:\\
\null\hspace{4ex}\verb+\renewcommand{\CVrightmargin}{150}+\\[-2ex]
\null\hspace{4ex}\verb+\renewcommand{\CVleftmargin}{50}+
\end{para}

\newpage
\begin{para}{Customize the font}
This class use\\[1ex]
\verb+\RequirePackage[sfdefault,lf]{carlito}+\\
to set the font. If you want to choose another one use the same \LaTeX\ commands and packages as usual or change these lines in the class file.
\end{para}

\begin{para}{Customize font size}
As for the length, here the font size (in points) to change:\\[1ex]
\begin{tabular}{lll}
\toprule
Length			& Default value	& Set	\\
\midrule
\verb+\CVfsnorm+	& 10		& Contents	\\
\verb+\CVfstit+		& 11		& Title	\\
\verb+\CVfssub+		& 12		& Subtitle	\\
\verb+\CVfshead+	& 16		& Header	\\
\bottomrule
\end{tabular}

\vspace{1ex}
And here the baseline skip (in points):\\[1ex]
\begin{tabular}{lcc}
\toprule
Length		& Default value	& Set	\\
\midrule
\verb+\CVbsnorm+	& 13		& Contents	\\
\verb+\CVbstit+	& 13.3		& Title	\\
\verb+\CVbssub+	& 14.4		& Subtitle	\\
\verb+\CVbshead+	& 19.2		& Header	\\
\bottomrule
\end{tabular}
\end{para}

\begin{para}{Panic?}
No, just remember: personal info commands, \texttt{para} environment, with or without title, \texttt{entry} command for your entries, and that's it!\\
If you find any bugs or issues, please contact me, and don't forget to bring a towel.\\[2ex]
\end{para}


\end{document}
